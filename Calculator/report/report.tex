%!Tex Program = xelatex
\documentclass[a4paper,12pt]{article}
\usepackage{amssymb,amsmath,amsfonts,amsthm}
%\usepackage{newtxtext,newtxmath}
\usepackage{fontspec,xunicode,xltxtra}
\usepackage[english]{babel}
\usepackage{listings}
\usepackage{mathrsfs}
%\usepackage{zhfontcfg}
%\usepackage{indentfirst}
%\usepackage[colorlinks,linkcolor=black]{hyperref}
%\usepackage{setspace}
%\usepackage{geometry}
% or whatever

%\usepackage[latin1]{inputenc}
% or whatever

\XeTeXlinebreaklocale "zh"
\XeTeXlinebreakskip = 0pt plus 1pt minus 0.1pt

% Ubuntu
\newfontfamily\hei{WenQuanYi Micro Hei Mono}
\newfontfamily\whei{WenQuanYi Zen Hei Mono}
\newfontfamily\kaishu{AR PL KaitiM GB}
\newfontfamily\song{AR PL SungtiL GB}
\newfontfamily\lishu{WenQuanYi Micro Hei Mono}
\setmainfont[Mapping=tex-text]{AR PL SungtiL GB}
\setsansfont[Mapping=tex-text]{AR PL KaitiM GB}
\setmonofont[Mapping=tex-text]{WenQuanYi Micro Hei Mono}

% Mac
%% \newfontfamily\hei{STHeitiSC-Light}
%% \newfontfamily\whei{STHeitiSC-Medium}
%% \newfontfamily\kaishu{STKaitiSC-Regular}
%% \newfontfamily\song{STSongti-SC-Regular}
%% \newfontfamily\lishu{STLibianSC-Regular}
%% \setmainfont[Mapping=tex-text]{STSongti-SC-Regular}
%% \setsansfont[Mapping=tex-text]{STKaitiSC-Regular}
%% \setmonofont[Mapping=tex-text]{STHeitiSC-Medium}
\renewcommand{\baselinestretch}{1.25}

\usepackage{fancyhdr}
\usepackage{lastpage}
\usepackage{ulem}
\usepackage{pgf}
\usepackage{graphicx}

\usepackage[hmargin={3.18cm, 3.18cm}, width=14.64cm,
             vmargin={2.54cm, 2.54cm}, height=24.62cm]{geometry}
\pagestyle{empty}

\usepackage{fancyhdr}
\usepackage{lastpage}
\pagestyle{fancy} %fancyhdr宏包新增的页面风格
\renewcommand{\headrulewidth}{0pt}
%\fancyhf{}
\cfoot{第 \thepage 页,共 \pageref{LastPage} 页}%当前页 of 总页数

\renewcommand\baselinestretch{1.2}
\setlength{\headwidth}{\textwidth}

%\renewcommand{\labelenumi}{\bf{\chinese{enumi}、}}
\renewcommand{\labelenumii}{\arabic{enumii}.}
\renewcommand{\labelenumiii}{\Roman{enumiii}}

\newcommand{\chuhao}{\fontsize{42pt}{\baselineskip}\selectfont} % 字号设置
\newcommand{\xiaochuhao}{\fontsize{36pt}{\baselineskip}\selectfont} % 字号设置
\newcommand{\yihao}{\fontsize{28pt}{\baselineskip}\selectfont} % 字号设置
\newcommand{\erhao}{\fontsize{21pt}{\baselineskip}\selectfont} % 字号设置
\newcommand{\xiaoerhao}{\fontsize{18pt}{\baselineskip}\selectfont} % 字号设置
\newcommand{\sanhao}{\fontsize{15.75pt}{\baselineskip}\selectfont} % 字号设置
\newcommand{\sihao}{\fontsize{14pt}{\baselineskip}\selectfont} % 字号设置
\newcommand{\xiaosihao}{\fontsize{12pt}{\baselineskip}\selectfont} % 字号设置
\newcommand{\wuhao}{\fontsize{10.5pt}{\baselineskip}\selectfont} % 字号设置
\newcommand{\xiaowuhao}{\fontsize{9pt}{\baselineskip}\selectfont} % 字号设置
\newcommand{\liuhao}{\fontsize{7.875pt}{\baselineskip}\selectfont} % 字号设置
\newcommand{\qihao}{\fontsize{5.25pt}{\baselineskip}\selectfont} % 字号设置

\newtheorem{theorem}{定理}
\newtheorem{definition}{定义}
\addto{\captionsenglish}{%
  \renewcommand{\refname}{参考文献}%
  \renewcommand{\proofname}{证明}%
  \renewcommand{\figurename}{图}%
%  \renewcommand{\bibname}{参考文献}%
}

\renewcommand{\theequation}{\arabic{section}.\arabic{equation}}
\renewcommand{\thetheorem}{\arabic{section}.\arabic{theorem}}
\renewcommand{\thedefinition}{\arabic{section}.\arabic{definition}}


\begin{document}

\title{\hei 计算器的实现设计思路和测试说明}


%\date{}
\maketitle

%%------------------------------------------
%% \section{问题}
%% \setcounter{equation}{0}
%% \setcounter{theorem}{0}
%% \setcounter{definition}{0}

\begin{enumerate}
\item 项目设计思路:

程序的实现主要分为以下几个步骤:
\begin{enumerate}
    \item {\hei 无效字符的过滤}:将输入的字符串除数字、小数点、加减乘除符号和括号外的符号过滤,形成新的字符串。这样可以方便之后的所有运算。
    \item {\hei 数字和操作符的分离}:在这一步中,我们希望最后能得到一个存储在\verb|vector<string>|中的中缀表达式。由于一个中缀表达式每个数和其他符号是分离的,因此我们需要在第一步过滤后的字符串中提取出数字(和小数点)作为一个整体的数存储,而不是一个个拆散的字符。本程序中通过在每个数字起点后截取第一个非数字(和小数点)前的字符串来实现。{\hei 特别地},对于负数前面的负号,程序通过判断条件将其和普通的减号区分开来,把负号看作数的一部分。
    \item {\hei 中缀到后缀的转换}:这一步骤是利用\verb|stack|实现的。具体的实现方法在课本中已经有非常详细的说明,本程序完成了其代码实现。值得注意的是,本程序使用了\verb|priLevel()|函数给每个符号赋予了不同的优先级,并通过给\verb|"("|最高的优先级和\verb|")"|最低的优先级来巧妙地解决了括号的输入输出问题。最后的结果存储在\verb|vector<string>|中。
    \item {\hei 后缀表达式的求值}:同样地,课本中也有对这一步详细的算法说明。每次遇到加减乘除符号时,程序都将栈顶的两个元素弹出并进行相应的运算。最后栈中存储的就是运算的结果。
    \item {\hei 中缀表达式合法性的判断}:{\hei 这一步在程序里并没有作为独立的过程实现,而是在第2-4步中对表达式的合法性进行判断}。下面将叙述程序是如何识别不合法的表达式的。
    \begin{enumerate}
        \item 数的不合法:数由小数点和数字构成。数的不合法主要是排除数内部出现多个小数点,以及小数点出现在数的首尾两端的情况。这些都在第二步截取数的时候进行了判断。
        \item 操作符(含括号,下同)之间的不合法:这一类的不合法现象主要涉及括号的匹配问题,以及多个操作符连续出现的问题。括号匹配在中缀转后缀的过程中得到判断(因为不合法的括号会找不到匹配的括号或者留在栈中)。多个操作符连续出现在第四步中得到判断。因为不合法的操作符会使得栈中没有足够的数进行运算。
        \item 操作符与数的结合不合法:这一类不合法现象主要有除数为0,以及左(右)括号的左(右)侧是数的不合法表达。它们分别在第四步和第二步得到了判断。
    \end{enumerate}
\end{enumerate}
\item 测试说明:
\begin{enumerate}
    \item 测试输入:
    \begin{enumerate}
        \item 运行 \verb|make| 命令,自动编译 \verb|main.cpp| 并生成可执行文件 \verb|test|;
        \item 运行 \verb|bash run| 命令即可;
        \item {\hei 测试数据的说明}:测试的数据都放在\verb|"input.txt"|中,总共有二十行(即二十个算数表达式)。其中前四行的数据与老师给出的输入输出样例对应,后十六行数据针对各种复杂的运算和各种可能报错的情况进行测试。{\hei 为了便于验算,表达式的长度都不算太长,测试者可以自行修改。}
    \end{enumerate}
    \item 测试输出:
        \begin{enumerate}
        \item 测试输出在\verb|"output.txt"|中,每一行对应着输入的一个结果。经验算,程序计算的结果都是正确的。
    \end{enumerate}
\end{enumerate}

\end{enumerate}

\bibliographystyle{plain}
\bibliography{ref}
\end{document}
