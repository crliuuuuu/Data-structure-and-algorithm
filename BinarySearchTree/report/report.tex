%!Tex Program = xelatex
\documentclass[a4paper,12pt]{article}
\usepackage{amssymb,amsmath,amsfonts,amsthm}
%\usepackage{newtxtext,newtxmath}
\usepackage{fontspec,xunicode,xltxtra}
\usepackage[english]{babel}
\usepackage{listings}
\usepackage{mathrsfs}
%\usepackage{zhfontcfg}
%\usepackage{indentfirst}
%\usepackage[colorlinks,linkcolor=black]{hyperref}
%\usepackage{setspace}
%\usepackage{geometry}
% or whatever

%\usepackage[latin1]{inputenc}
% or whatever

\XeTeXlinebreaklocale "zh"
\XeTeXlinebreakskip = 0pt plus 1pt minus 0.1pt

% Ubuntu
\newfontfamily\hei{WenQuanYi Micro Hei Mono}
\newfontfamily\whei{WenQuanYi Zen Hei Mono}
\newfontfamily\kaishu{AR PL KaitiM GB}
\newfontfamily\song{AR PL SungtiL GB}
\newfontfamily\lishu{WenQuanYi Micro Hei Mono}
\setmainfont[Mapping=tex-text]{AR PL SungtiL GB}
\setsansfont[Mapping=tex-text]{AR PL KaitiM GB}
\setmonofont[Mapping=tex-text]{WenQuanYi Micro Hei Mono}

% Mac
%% \newfontfamily\hei{STHeitiSC-Light}
%% \newfontfamily\whei{STHeitiSC-Medium}
%% \newfontfamily\kaishu{STKaitiSC-Regular}
%% \newfontfamily\song{STSongti-SC-Regular}
%% \newfontfamily\lishu{STLibianSC-Regular}
%% \setmainfont[Mapping=tex-text]{STSongti-SC-Regular}
%% \setsansfont[Mapping=tex-text]{STKaitiSC-Regular}
%% \setmonofont[Mapping=tex-text]{STHeitiSC-Medium}
\renewcommand{\baselinestretch}{1.25}

\usepackage{fancyhdr}
\usepackage{lastpage}
\usepackage{ulem}
\usepackage{pgf}
\usepackage{graphicx}

\usepackage[hmargin={3.18cm, 3.18cm}, width=14.64cm,
             vmargin={2.54cm, 2.54cm}, height=24.62cm]{geometry}
\pagestyle{empty}

\usepackage{fancyhdr}
\usepackage{lastpage}
\pagestyle{fancy} %fancyhdr宏包新增的页面风格
\renewcommand{\headrulewidth}{0pt}
%\fancyhf{}
\cfoot{第 \thepage 页,共 \pageref{LastPage} 页}%当前页 of 总页数

\renewcommand\baselinestretch{1.2}
\setlength{\headwidth}{\textwidth}

%\renewcommand{\labelenumi}{\bf{\chinese{enumi}、}}
\renewcommand{\labelenumii}{\arabic{enumii}.}
\renewcommand{\labelenumiii}{\Roman{enumiii}}

\newcommand{\chuhao}{\fontsize{42pt}{\baselineskip}\selectfont} % 字号设置
\newcommand{\xiaochuhao}{\fontsize{36pt}{\baselineskip}\selectfont} % 字号设置
\newcommand{\yihao}{\fontsize{28pt}{\baselineskip}\selectfont} % 字号设置
\newcommand{\erhao}{\fontsize{21pt}{\baselineskip}\selectfont} % 字号设置
\newcommand{\xiaoerhao}{\fontsize{18pt}{\baselineskip}\selectfont} % 字号设置
\newcommand{\sanhao}{\fontsize{15.75pt}{\baselineskip}\selectfont} % 字号设置
\newcommand{\sihao}{\fontsize{14pt}{\baselineskip}\selectfont} % 字号设置
\newcommand{\xiaosihao}{\fontsize{12pt}{\baselineskip}\selectfont} % 字号设置
\newcommand{\wuhao}{\fontsize{10.5pt}{\baselineskip}\selectfont} % 字号设置
\newcommand{\xiaowuhao}{\fontsize{9pt}{\baselineskip}\selectfont} % 字号设置
\newcommand{\liuhao}{\fontsize{7.875pt}{\baselineskip}\selectfont} % 字号设置
\newcommand{\qihao}{\fontsize{5.25pt}{\baselineskip}\selectfont} % 字号设置

\newtheorem{theorem}{定理}
\newtheorem{definition}{定义}
\addto{\captionsenglish}{%
  \renewcommand{\refname}{参考文献}%
  \renewcommand{\proofname}{证明}%
  \renewcommand{\figurename}{图}%
%  \renewcommand{\bibname}{参考文献}%
}

\renewcommand{\theequation}{\arabic{section}.\arabic{equation}}
\renewcommand{\thetheorem}{\arabic{section}.\arabic{theorem}}
\renewcommand{\thedefinition}{\arabic{section}.\arabic{definition}}


\begin{document}

\title{\hei 二叉搜索树排序设计思路和测试说明}


%\date{}
\maketitle

%%------------------------------------------
%% \section{问题}
%% \setcounter{equation}{0}
%% \setcounter{theorem}{0}
%% \setcounter{definition}{0}

\begin{enumerate}
\item 项目设计思路:
\begin{enumerate}
    \item \verb|BinarySearchTree.h|的实现:
    \begin{enumerate}
        \item \verb|BinarySearchTree.h|的框架摘自课本\verb|4.3|节的二叉查找树的代码。在能实现二叉搜索树排序的前提下,我略去了头文件中与本项目无关的类成员函数。此外我改写了\verb|printTree()|函数,使得\verb|main.cpp|调用该函数后即可将排序完成的数据存储在\verb|vector|中。因为二叉搜索树满足\verb|:|对于所有节点\verb|X|,它左子树的所有项均小于\verb|X|的项,右子树所有项均大于\verb|X|中的项。所以只需对二叉搜索树进行中序遍历,并将每个节点用\verb|push_back()|操作推入\verb|vector|的尾端即可。
    \end{enumerate}
    \item \verb|main.cpp|的实现:
    \begin{enumerate}
        \item {\hei 数据的随机生成}:用户输入进行排序的元素个数\verb|N|后,程序将自动生成一个大小为\verb|N|的 \verb|vector<TN> DATA|。由于本程序使用的具体的模板类型是\verb|Int|,因此,为了方便验证排序的结果,\verb|DATA|将恰好含有\verb|[0,N-1]|中的全部整数(这样排序好的数据恰好是\verb|0,1,|
        \verb|...N-1|
        ,{\hei 可以方便我们的检验})。具体的实现方式是,对于\verb|[0,|
        \verb|N-1]|中的全部整数,给其一个随机的下标\verb|p|。如果\verb|vector[p]|已经被赋值,则向下寻找至第一个未赋值的单元并给其赋值。
        \item {\hei \verb|Randomized_BST_sort()|的实现}:根据项目要求,我们首先对\verb|DATA|中的数据进行打乱。实现方式与随机生成几乎一模一样,只需给每个数赋予一个新的随机下标即可。然后,将打乱的数据按顺序插入二叉搜索树中,并调用\verb|printTree()|把排序后的结果覆盖到\verb|DATA|中以便之后的输出。
    \end{enumerate}
    \item {\hei 算法的时间复杂度}:
    
    二叉搜索树排序主要涉及树的建立和中序遍历两个步骤,具体分析如下:
    \begin{enumerate}
        \item 二叉搜索树的建立:对于一个有\verb|N|个节点的二叉树,根据课本
        \verb|4.3.6|的分析,任意节点的期望深度是\verb|O(log N)(以2为底,下同)|。对于每个节点,在期望深度下插入到树中,因此这一步的时间复杂度是\verb|O(N log N)|。
        \item 中序遍历:设\verb|N|个节点遍历所需时间为\verb|T(N)|,则根据\verb|printTree()|
        的代码,可得:\verb|T(N)=2*T(N/2)+1|。该递推式满足主定理的第一种情况,由此得\verb|T(N)=θ(N)|。
        \item 综上,理论上二叉搜索树排序的时间效率是\verb|O(N log N)|,为了验证这个结论,程序对随机生成的\verb|DATA|进行多次打乱\verb|(默认200次)|,并调用\verb|high_resolution_clock|对整个排序操作进行计时,取平均值后输出。
    \end{enumerate}
\end{enumerate}
\item 测试说明:
\begin{enumerate}
    \item 测试输入:
    \begin{enumerate}
        \item 运行 \verb|make| 命令,自动编译 \verb|main.cpp| 并生成可执行文件 \verb|test|;
        \item 运行 \verb|bash run| 命令,{\hei 在同一行后输入一个正整数\verb|N|,表示进行排序的元素的个数}。
        \item {\hei 注意事项:建议\verb|N|的值不要太小也不要太大}。如果\verb|N|的值太小,程序运行过快。由于排序运行的平均时间是以微秒为单位的,可能会输出\verb|0 microseconds|的结果;如果\verb|N|的值过大,程序运行会十分缓慢。{\hei 建议的N的值是50-50000。}
    \end{enumerate}
    \item 测试输出:
        \begin{enumerate}
        \item 测试输出打乱排序前随机生成的数组\verb|(恰好含有[0,N-1]中的全部整数)|、排序后的数组和二叉搜索树排序所用平均时间\verb|T(N)|。具体在输出中都有详细的说明。
    \end{enumerate}
\end{enumerate}
\item 测试结论:
\begin{enumerate}
     \item 经验证,无论\verb|N|取什么值,输出的排序后的\verb|DATA|都是\verb|0,1,...N-1|,这说明{\hei 该二叉搜索树的排序算法是正确的}。
     \item 对于二叉搜索树排序所用的平均时间\verb|T(N)|,经多次测试得如下平均结果\verb|(|部分值\verb|)|:
     
     \begin{center}
     \begin{tabular}{|c|c|}
     \hline
        \verb|N|  & \verb|T(N)/ms| \\
     \hline
     100  & 0.006 \\
     \hline
     1000 & 0.12 \\
     \hline
     10000 & 1.3 \\
     \hline
     100000 & 23 \\
     \hline
     \end{tabular}
     \end{center}

     经过计算,可以发现当\verb|N|较小\verb|(<500)|或者过大\verb|(>10000)|时,\verb|T(N)|的增长速率要大于\verb|(N log N)|。{\hei 对此的解释是},虽然理论上说二叉搜索树排序的运行时间是\verb|O(N log N)|,但是它有一个明显的问题,即在中序遍历的时候用到了线性附加内存。对于\verb|printTree()|中的\verb|push_back()|操作,正如在\verb|Vector|模板类的实现中所分析的,
     \verb|SPARE_CAPACITY|不能过大或者过小。因此。当\verb|N|过大时,不断进行\verb|push_back()|操作都会明显地减慢排序的速度;当\verb|N|过小时,进行\verb|push_back()|操作时间在总排序的时间中占比较大,而且此时的二叉查找树不平衡的概率较大\verb|(|因为节点数较少\verb|)|。这些因素从而使得\verb|T(N)|的增长速率要大于\verb|(N log N)|。
     
     为了验证该想法的正确性,我将\verb|BinarySearchTree.h|头文件中的\verb|push_back()|操作改为直接输出至屏幕\verb|(|这样能保证对不同\verb|N|的输出操作时间几乎都是相同的\verb|)|,经多次测试得到如下平均结果\verb|(|部分值\verb|)|:

     \begin{center}
     \begin{tabular}{|c|c|}
     \hline
        \verb|N|  & \verb|T(N)/ms| \\
     \hline
     100  & 0.59 \\
     \hline
     1000 & 6.8 \\
     \hline
     10000 & 72 \\
     \hline
     100000 & 780 \\
     \hline
     \end{tabular}
     \end{center}
     
     由此可见,虽然\verb|T(N)|增大了不少\verb|(|因为输出的用时较长\verb|)|,但是\verb|T(N)|的增长速率要小于\verb|(N log N)|。由此验证了根据理论得出的结论:平均情况下,二叉搜索树排序的时间复杂度是\verb|O(N log N)|。

\end{enumerate}
\end{enumerate}

\bibliographystyle{plain}
\bibliography{ref}
\end{document}
